% !TEX root = ../Projektdokumentation.tex
\section{Projektplanung} 
\label{sec:Projektplanung}


\subsection{Projektphasen}
\label{sec:Projektphasen}

\begin{itemize}
	\item Für diese Umsetzung des Projektes standen dem Auszubildenen 80 Stunden zur Verfügung.
	Welche vor Projektbeginn in verschiedene Phasen aufgeteilt wurden. Diese Phasen sind aus der Tabelle 1: Zeitplanung zu entnehmen
\end{itemize}

Tabelle~\ref{tab:Zeitplanung}
\tabelle{Zeitplanung}{tab:Zeitplanung}{ZeitplanungKurz}\\
Ein Detaillierter Zeitplan mit einzelnen Teilaufgaben kann in \Anhang{app:Zeitplanung} eingesehen werden.


\subsection{Ressourcenplanung}
\label{sec:Ressourcenplanung}

\begin{itemize}
	\item Um die Planung der erforderlichen Ressourcen effizient zu gestalten, wurde gezielt auf bereits vorhandene Infrastruktur zurückgegriffen. Die Auswahl der Hardware und Software erfolgte so, dass keine zusätzlichen Anschaffungen erforderlich waren. Hardware war vollständig vorhanden, da alle Mitarbeiter über einen ausgestatteten Büroarbeitsplatz mit Standard-Peripherie verfügten. Die Softwareauswahl konzentrierte sich auf Open-Source-Tools wie PostgreSQL und Python sowie bereits lizenzierten Programmen (Business Central API), die während der Projektlaufzeit genutzt wurden, um die Datenverarbeitung sicherzustellen. Zusätzlich wurde die Personaleinsatzplanung unter Berücksichtigung der verschiedenen Rollen und Verantwortlichkeiten organisiert. Der Auszubildende übernahm die Hauptverantwortung in der Entwicklung, was insgesamt 80 Stunden veranschlagte. Ein Entwickler wurde für 10 Stunden eingeplant, um gezielt bei technischen Fragen zu unterstützen. Die SCM-Abteilung war schließlich dafür verantwortlich, die Anforderungen zu formulieren und das System während der Testphase auf seine Praxistauglichkeit zu überprüfen. Die Kombination aus vorhandenen Ressourcen und gezielter, bedarfsorientierter Planung sorgte dafür, dass die Projektkosten gering blieben und die notwendige Unterstützung jederzeit gewährleistet war.
\end{itemize}


\subsection{Entwicklungsprozess}
\label{sec:Entwicklungsprozess}
\begin{itemize}
	\item Für die Entwicklung dieses Projekts wurde das Wasserfallmodell als Vorgehensmodell gewählt. Das Wasserfallmodell zeichnet sich durch eine klare und lineare Abfolge von Phasen aus, was insbesondere bei Projekten mit fest definierten Anforderungen und einem strukturierten Ablauf von Vorteil ist. Da die Anforderungen an das System im Vorfeld detailliert analysiert und dokumentiert wurden, ermöglichte das Wasserfallmodell eine schrittweise und systematische Umsetzung der Projektphasen von der Anforderungsanalyse über die Implementierung bis hin zur abschließenden Testphase. 
	\item Die Entscheidung für dieses Modell wurde zudem getroffen, da die festen Meilensteine eine effiziente Ressourcenplanung erleichtern und sicherstellen, dass jede Phase abgeschlossen und überprüft wird, bevor die nächste beginnt. Dies gewährleistet eine hohe Kontrolle über den Projektverlauf und minimiert das Risiko unvorhergesehener Änderungen im Entwicklungsprozess, wodurch die Qualität und Konsistenz des Endprodukts gefördert werden.
\end{itemize}
