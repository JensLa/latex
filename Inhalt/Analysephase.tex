% !TEX root = ../Projektdokumentation.tex
\section{Analysephase} 
\label{sec:Analysephase}


\subsection{Ist-Analyse} 
\label{sec:IstAnalyse}
\begin{itemize}
	\item Derzeit werden täglich alle Warensendungen manuell geprüft, indem Lieferungen mit Trackingnummern herausgefiltert und im Onlineportal des externen Versanddienstleisters überprüft werden. Mitarbeiter in der Abteilung Supply Chain Management (SCM) durchforsten Verkaufsaufträge nach Trackingnummern und prüfen im Versandportal den aktuellen Status der Sendungen. Der Fokus liegt hierbei darauf, Unregelmäßigkeiten wie Verzögerungen oder Fehlzustellungen zu identifizieren. Bei auffälligen Zuständen wird die zuständige Abteilung informiert, um die erforderlichen Maßnahmen einzuleiten. 
	
\end{itemize}

\subsection{Wirtschaftlichkeitsanalyse}
\label{sec:Wirtschaftlichkeitsanalyse}
\begin{itemize}
	\item Die Wirtschaftlichkeit des Projekts ließ sich unkompliziert anhand der eingesparten Arbeitszeit der SCM-Mitarbeiter berechnen, da diese durch die Automatisierung des Prüfungsprozesses deutlich reduziert wird.
	\item Hierzu wurde im Anhang eine genaue Wirtschaftlichkeitsberechnung durchgeführt.
	\item verweiss auf Anhang ?  (siehe \ref{sec:WirtschaftlichkeitsberechnungAnhang})
\end{itemize}



\subsubsection{Amortisationsdauer}
\label{sec:Amortisationsdauer}
\begin{itemize}
	\item Der größte Vorteil ist die drastische Reduzierung des manuellen Arbeitsaufwands in der SCM-Abteilung. Zuvor wurden rund 189 Stunden jährlich für die manuelle Überprüfung von Sendungen benötigt. Durch die Automatisierung wird diese Aufgabe ohne manuelle Eingriffe erledigt, was zu jährlichen Einsparungen von rund 5.670 € an Arbeitskosten führt. Diese Einsparungen ermöglichen eine Amortisation der Projektkosten in sehr kurzer Zeit, etwa innerhalb von zwei Monaten nach Einführung.
	\item Die Betriebskosten des Systems belaufen sich auf nur 264 € jährlich, die größtenteils aus gelegentlichem Support bestehen. Durch diese geringen laufenden Kosten, kombiniert mit den hohen Einsparungen, erreicht das Projekt bereits im ersten Jahr eine vollständige Amortisation und generiert danach ausschließlich Einsparungen.
	
\end{itemize}

\paragraph{Beispielrechnung (verkürzt)}
Bei einer Zeiteinsparung von 10 Minuten am Tag für jeden der 25 Anwender und 220 Arbeitstagen im Jahr ergibt sich eine gesamte Zeiteinsparung von 
\begin{eqnarray}
25 \cdot 220 \mbox{ Tage/Jahr} \cdot 10 \mbox{ min/Tag} = 55000 \mbox{ min/Jahr} \approx 917 \mbox{ h/Jahr} 
\end{eqnarray}

Dadurch ergibt sich eine jährliche Einsparung von 
\begin{eqnarray}
917 \mbox{h} \cdot \eur{(25 + 15)}{\mbox{/h}} = \eur{36680}
\end{eqnarray}

Die Amortisationszeit beträgt also $\frac{\eur{2739,20}}{\eur{36680}\mbox{/Jahr}} \approx 0,07 \mbox{ Jahre} \approx 4 \mbox{ Wochen}$.


\subsection{Nutzwertanalyse}
\label{sec:Nutzwertanalyse}
\begin{itemize}
	\item Darstellung des nicht-monetären Nutzens (\zB Vorher-/Nachher-Vergleich anhand eines Wirtschaftlichkeitskoeffizienten). 
\end{itemize}

\paragraph{Beispiel}
Ein Beispiel für eine Entscheidungsmatrix findet sich in Kapitel~\ref{sec:Architekturdesign}: \nameref{sec:Architekturdesign}.


\subsection{Anwendungsfälle}
\label{sec:Anwendungsfaelle}
\begin{itemize}
	\item Welche Anwendungsfälle soll das Projekt abdecken?
	\item Einer oder mehrere interessante (!) Anwendungsfälle könnten exemplarisch durch ein Aktivitätsdiagramm oder eine \ac{EPK} detailliert beschrieben werden. 
\end{itemize}

\paragraph{Beispiel}
Ein Beispiel für ein Use Case-Diagramm findet sich im \Anhang{app:UseCase}.


\subsection{Qualitätsanforderungen}
\label{sec:Qualitaetsanforderungen}
\begin{itemize}
	\item Welche Qualitätsanforderungen werden an die Anwendung gestellt (\zB hinsichtlich Performance, Usability, Effizienz \etc (siehe \citet{ISO9126}))?
\end{itemize}


\subsection{Lastenheft/Fachkonzept}
\label{sec:Lastenheft}
\begin{itemize}
	\item Auszüge aus dem Lastenheft/Fachkonzept, wenn es im Rahmen des Projekts erstellt wurde.
	\item Mögliche Inhalte: Funktionen des Programms (Muss/Soll/Wunsch), User Stories, Benutzerrollen
\end{itemize}

\paragraph{Beispiel}
Ein Beispiel für ein Lastenheft findet sich im \Anhang{app:Lastenheft}. 
